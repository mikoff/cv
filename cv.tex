\documentclass{tccv}
\usepackage[english]{babel}
 %% Option 'familydefault' only if the base font of the document is to be sans serif
\usepackage{fancyhdr}
\usepackage{lastpage}
%\pagestyle{fancy}
\usepackage{supertabular}
\usepackage{enumitem}
\usepackage{fontawesome5}
\usepackage{fontspec}
\usepackage[T1]{fontenc}
\usepackage[light,familydefault=true]{Chivo}
\usepackage{textcomp}
\usepackage{xcolor}



%\usepackage{geometry}
% \geometry{a4paper, landscape, bmargin=20mm}
%\geometry{a4paper, bmargin=20mm}

\addtokomafont{subsection}{\centering\Large}
\usepackage{xpatch}
\makeatletter
\newcommand{\about@me}{}
\newcommand{\aboutme}[1]{\renewcommand{\about@me}{\subsection{About Me}#1}}
\xpatchcmd{\part}{\end{center}}{\end{center}\about@me\vspace*{1cm}}
\makeatother

\title{Aleksandr Mikov Resume}

\cfoot{}
\rfoot{\thepage\ of \pageref*{LastPage}, \today}
\renewcommand{\headrulewidth}{0.0pt}
\renewcommand{\footrulewidth}{0.4pt}

\definecolor{blue(ncs)}{rgb}{0.0, 0.53, 0.74}
\definecolor{olivegreen}{rgb}{0.52, 0.52, 0.1}
\definecolor{forestgreen}{rgb}{0.133, 0.545, 0.133}

\begin{document}

\aboutme{\center{\textcolor{forestgreen}{\textbf{PhD}}, Researcher and algorithm developer in the indoor and outdoor navigation field\\with an emphasis on automotive and pedestrian applications.}}
\part{Aleksandr Mikov}


\section{Employment}

\begin{eventlist}
\event{2022 -- now}
     {\textcolor{red}{B}osch, XC/AS, Stuttgart}
     {Senior Researcher}
\begin{itemize}
     \item Developed localization algorithms for parking lots \& underground garages based on odometry and perception data
     \item Prototyped mapping algorithm for parking lots \& underground garages
\end{itemize}

\event{2020 -- 2022}
     {\textcolor{red}{H}uawei, St. Petersburg}
     {Lead Researcher}
\begin{itemize}
     \item Worked on Intelligent Transport Systems problems: radar-camera calibration, pose estimation, multi-target tracking
     \item Developed unsupervised alignment algorithm for radar-camera pairs deployed at the intersection
\end{itemize}

\event{2018 -- 2020}
     {\textcolor{blue(ncs)}{N}avigine Inc. \& \textcolor{red}{Y}andex, Skolkovo \& Moscow}
     {Researcher \& Senior Software Developer}
\begin{itemize}
    \item Developed Sensor Fusion of GNSS, Inertial and Visual Odometry data. The position error after 5 minutes of autonomous positioning didn't exceed 1\% of total travelled distance (See  \href{https://www.dropbox.com/s/kaij23xphdqjv5j/kalman_heading_correction_on.html?dl=1}{Example})
    \item Worked on Tightly Coupled GNSS/INS Sensor Fusion toolbox
    \item Developed sensor fusion algorithms for automotive positioning using GPS, odometer and Analog Devices inertial sensors. Enabled fast and efficient cold start without GPS (See \href{https://www.dropbox.com/s/yjag8qm7ax6skoa/vehicle_positioning_demo.pptx}{Results})
    \item Developed algorithms for automotive positioning in GPS-denied environment using CAN-data(steering wheel angle, odometer) and road information. With the proposed algorithms car is able to robustly estimate its position hours after GPS signal loss (See \href{https://www.dropbox.com/s/awil97l95az982h/AutotomotiveCanFusion.png}{Demo})
    \item Accelerated the development of localization library  which fuses Bluetooth, WiFi, GPS and inertial measurements, strengthened it to provide even more robust navigation solution
    \item Implemented and integrated indoor Particle Filter based positioning algorithms for Yandex Maps
    \item Developed Sensor Fusion solution of GNSS, Inertial and Odometer data for Yandex Drive car positioning using low-cost sensors. The solution able to work robustly in GNSSs-denied and GNSS-spoofed environments (See \href{https://www.dropbox.com/s/jkawevakrhy58hc/LefortovskiyFast.mp4}{Video})
\end{itemize}

\begin{figure}[t]
\personal
    []
    {Stuttgart, Germany}
    {+49 176 557 30340}
    {} %sasha.mikoff@gmail.com
    {linkedin.com/in/sasha-mikoff}
    {mikoff.github.io}
    {scholar.google.com/citations?user=skjYlgkAAAAJ}
    {today}
\end{figure}

\event{2012 -- 2018}
     {OOO NANOSETI, Petrozavodsk}
     {Researcher \& Software Developer}
     
Developed sensor fusion algorithms, motion classifiers and event-detectors, applied various optimization techniques.
Worked in an excellent team of professionals.

\begin{itemize}
    \item Constructed the models and algorithms for an accurate IMU calibration. The algorithms cure sensors imperfections: scale, non-orthogonality, bias and triads' misalignment
    \item Developed high-accurate sensor fusion algorithms for automotive positioning using only inertial sensors (See \href{https://www.youtube.com/watch?v=w9zKA05Dwww}{Demo})
    \item Developed the dead-reckoning algorithms for pedestrians (See \href{https://www.youtube.com/watch?v=ge9EXeF7SJs}{Demo 1}, \href{https://www.youtube.com/watch?v=GqNHD0K2BRI}{Demo 2})
    \item Designed motion activity classifier using Machine Learning techniques. It is able to detect walking, running, walking upstairs, walking downstairs and zero motion
    \item Intensively worked with Kalman filters, Bayesian methods and probability, statistical signal processing, multiple hypothesis tests
    \item Designed an embedded firmware for mobile hand-
held devices connected in wireless sensor network based on UWB/BLE with a focus on their self localization, voice transmission and sensor fusion
    \item Tutored students on inertial navigation topics and prepared lab materials
\end{itemize}

\newpage
\event{2011 -- 2012}
     {Budget Monitoring Center, Petrozavodsk}
     {Software developer}
\begin{itemize}
    \item Composed and proposed the statistical metrics to estimate the efficiency of local government services
    \item Deployed database mining algorithms and analytical services
\end{itemize}

\end{eventlist}

% \newpage
\section{Academic Career}

\begin{yearlist}

\item[Advisor: \href{https://petrsu.ru/persons/322/voronov}{Roman Voronov}]{Dec. 2021}
     {PhD \textnormal{in Mathematical Modeling, Numerical Methods and Software Engineering} \newline Thesis title: \textnormal{Automotive dead reckoning algorithms for vehicles using low-cost MEMS IMUs}}
     {Petrozavodsk State University}
     

\item[With honors \newline GPA: 5.0 \newline
      Advisor: \href{http://lab127.karelia.ru/~alexmou/resume_alexmou_eng.pdf}{Alexey Moschevikin} \newline
      Co-advisor: \href{https://ei.hs-offenburg.de/nc/ansprechpartner/personen-details-lsf-cache/lsf/704/6/1127/} {Axel Sikora}]{June 2013}
     {M.S. in Computer Science 
      \newline 
      M.S. in Computer Engineering}
     {Petrozavodsk State University}
\end{yearlist}
\begin{yearlist}
\item[With honors \newline GPA: 4.77 \newline
      Advisor: \href{http://openbudgetrf.ru/sotrudnik-publications/4/}{Ilya Pennie}]{June 2011}
     {B.S. in Computer Science 
      \newline 
      B.S. in Computer Engineering}
     {Petrozavodsk State University}

\end{yearlist}

\section{Courses}

\begin{yearlist}

\item[\href{https://www.dropbox.com/s/116ku7br2kqn5nr/doc00422820191224182244.pdf}{Certificate}]{2019}
     {Algorithms for integrated inertial satellite navigation systems}
     {Lomonosov Moscow State University, MSU}

\item[\href{https://www.coursera.org/account/accomplishments/records/PJCQYC92YUSU}{Coursera Certificate}]{2017}
     {Machine Learning: Statistics for data analysis}
     {Moscow Institute of Physics and Technology \& Yandex}

\item[\href{https://www.coursera.org/account/accomplishments/records/2MZW385H2Y2U}{Coursera Certificate}]{2016}
     {Machine Learning: Unsupervised Learning}
     {Moscow Institute of Physics and Technology \& Yandex}
     
\end{yearlist}
\begin{yearlist}

\item[\href{https://www.coursera.org/account/accomplishments/records/UFDJLATYB83T}{Coursera Certificate}]{2016}
     {Machine Learning: Supervised Learning}
     {Moscow Institute of Physics and Technology \& Yandex}

\item[\href{https://www.coursera.org/account/accomplishments/records/PJCQYC92YUSU}{Coursera Certificate}]{2016}
     {Math and Python for data analysis}
     {Moscow Institute of Physics and Technology \& Yandex}

\item[]{2016}
     {Robotics: Aerial Robotics}
     {University of Pennsylvania on Coursera}

\item[]{2015}
     {Digital Signal Processing}
     {Faculty of Extension Courses, PetrSU}

\item[]{2014}
     {Fundamentals of business: information services}
     {Faculty of Extension Courses, PetrSU}

\end{yearlist}
\begin{yearlist}

\item[]{2013}
     {Computer networks and communications}
     {Faculty of Extension Courses, PetrSU}

\end{yearlist}

\section{Awards and acknowledgments}

\begin{yearlist}
\item {2015}
     {Intel Hackaton}
     {The member of the winning team of IoT
2015 hackaton. Presented a device which was able to measure the physical characteristics of the environment and to make the recommendations about work schedule}

\item{2014}
     {The winner of <<Youth Scientific Innovation
Competition>> (UMNIK)}
     {My funding proposal for <<The
development of inertial navigation system for pedestrians
in GPS-denied environment>> got a \$10000 grant}

\end{yearlist}

\begin{yearlist}
\item{2013}
     {First prize EvAAL-2013}
     {The RealTrac positioning system, which I contributed to, took the frist place in International competition <<Evaluating AAL Systems through Competitive Benchmarking, EvAAL-2013>>}

\item{2013}
     {DAAD Scholarship, Germany}
     {Research work at Offenburg University of Applied Sciences}

\item{2013}
     {Personal Scholarship}
     {The winner of the scholarship from the head of the republic of Karelia}

\item{2012}
     {Personal Scholarship}
     {The winner of the special scholarship for young researchers from the Russian Parliament}

\item{2012}
     {Microsoft Student Partner}
     {Microsoft Student Partner in Petrozavodsk State University in 2011-2012}
\end{yearlist}

\newpage
\onecolumn
\section{Skills}

{\large{\textbf{Software development}}}
\vspace{1em}

\begin{supertabular}{ p{4.0cm} | p{12.5cm} }
  Platforms & Proficiency in Linux and its tools. Special exposure to real-time operating systems (FreeRTOS). Able to develop on Unix or Microsoft desktop systems. Solid background in embedded programming on low-powered microcontrollers (ARM Cortex based: STM32F4 and nRF51 families). \\\hline
  IDEs & Experience with Visual Studio Code, VIM.  \\\hline
  Compilers & GNU Compiler Collection, GNU Arm Embedded Toolchain.  \\\hline
  Hardware & CAN, I2C, I2S, SAI, SPI, DMA, inertial sensors, nanoLOC, BLE, UWB, Audio Codecs, SocketCAN\\\hline
  SCM & GIT, SVN. \\
\end{supertabular}

\vspace{2em}
{\large{\textbf{Programming Technologies}}}
\vspace{1em}

\begin{supertabular}{ p{4.0cm} | p{12.5cm} }
  C/C++ & 8 years of experience. Good understanding of program flow, memory management and real-time systems behaviour. Solid knowledge of standard (STL) and scientific (Eigen) libraries. \\\hline
  Python & 6 years of experience. Used for research, algorithm development and verification. Gained the competence in Numpy, Scipy, Pandas and Scikit-Learn. A lot of experience with visualizations and data representation using Jupyter tools.  \\\hline
  Matlab & Good proficiency. Used before Python for various research problems.  \\\hline
  \LaTeX & Used for paper publishing, scientific reports and professional documentation. \\
\end{supertabular}

\vspace{2em}
{\large{\textbf{Sciencific knowledge}}}
\vspace{1em}

\begin{supertabular}{ p{4.0cm} | p{12.5cm} }
  Computer science & Functional and object-oriented programming, data structures, algorithm design and its complexity estimation, Montecarlo simulations, software engineering, debugging and code profiling, subversioning \\\hline
  Machine learning & Classification, clusterization, linear regression, statistics, Particle Filters, Bayesian methods, probability theory, learning methods.  \\\hline
  Signal processing & Kalman filters, sensor calibration, smoothing, dimensionality reduction, sampling, numerical computing, linear algebra, least squares, minimization and maximization problems.  \\\hline
  Navigation systems & GPS, GLONASS, GALILEO, Inertial Sensors, odometers, motion constraints, Ackermann model, Map matching, Factor Graphs. \\
\end{supertabular}

\section{Publications}
Full list of publications is available \href{https://scholar.google.ru/citations?user=skjYlgkAAAAJ}{here}

\end{document}
